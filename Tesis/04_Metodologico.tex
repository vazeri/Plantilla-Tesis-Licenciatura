% Capítulo III Marco Metodológico

% Este capitulo cubre la descripción de la metodología, técnicas y herramientas con las cuales se resolverá el o los problemas de esta TESIS (¿con que se va a hacer?)

\chapter{ MARCO METODOLÓGICO }

\section{ Metodología del Sistema de Información  }

\textit{En este punto deben indicarse los métodos, las técnicas y herramientas, que se utilizarán para llevar a cabo la investigación.}

\section{ Hardware }

\textit{En caso de utilizar componentes electrónicos físicos deben describirse de manera breve y concisa.}

\subsection{ Sensor de corriente }

El transformador sensor de corriente Yhdc es fabricado en Beijing YaoHuadechang por Electronic Co., Ltd el modeo, SCT-013-000, con numero de parte SKU THM105C4B. Sera utilizado debido a su capacidad de sensar hasta 100Amperes, cubriendo así los requisitos mínimos de la NOM001 y previendo la instalación del dispositivo en sistemas de mayor consumo de corriente.

\section{ Lenguajes de programación e Internet }

\textit{Para el caso de los lenguajes de programación de igual manera que el hardware se mencionan solo los aspectos mas relevantes de manera breve y concisa. }

\subsection { Hyper Text Markup Language (HTML) }

\textit{Descripción breve de HTML utilizando el sistema de citas (todo lo que se copie y pegue debe estar citado) de otra manera se le considera plagio.}

Lenguaje de marcas de hipertexto, lenguaje utilizado para la elaboración de páginas web es un estándar que sirve de referencia para la elaboración de páginas web en sus diferentes versiones a través de etiquetas, define una estructura básica y un código para la definición de contenido de una página web. Es un estándar a cargo de la $W3C$, organización dedicada a la estandarización de tecnologías ligadas a la web. \cite{html}.

\section{ Resumen del Capitulo 4 }

\textit{Resumen del capitulo e introducción al siguiente capitulo.}

% Capítulo III Marco Metodológico