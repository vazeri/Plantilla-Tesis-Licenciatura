%%%%%%%%%%%%%%%%%%%%%%%%%%%%%%%%%%%%%%%%%%%%%%%%%%%%%%%%%%%%%%%%%%%%%%%%%%%%%%%%%%%%%
% Protocolo de la tesis
%%%%%%%%%%%%%%%%%%%%%%%%%%%%%%%%%%%%%%%%%%%%%%%%%%%%%%%%%%%%%%%%%%%%%%%%%%%%%%%%%%%%%

% El protocolo es la parte mas importante para los asesores, ya que con esta es posible que entiendan la totalidad del trabajo en un par de paginas, si necesidad de leer todo el trabajo, en caso de querer ahondar mas en el trabajo se dirigirán al índice en donde solo con revisarlo deberán entender a grandes rasgos el como se desarrollo el trabajo de tesis 

\newpage %Termina la pagina actual y desplaza el contenido a una nueva pagina
\newpage


\newpage
~\vfill
%\thispagestyle{empty}

\emph{DEDICATORIA}
\hfill \break

La dedicatoria es un breve texto, que se escribe al principio  o al final de un libro, tesis, película, canción, etc., para expresar agradecimiento por el apoyo o ayuda en el desarrollo y finalización de la obra, sea del género  o formato que fuere esta. Dirigidas hacia una persona, personas o instituciones  a las que se agradece por su ayuda o apoyo.   

La palabra tesis significa "proposición o conclusión que se mantiene por razonamiento”. Se entiende por tesis a la  disertación escrita que se presenta para la obtención de un título, licenciatura, doctorado, o como conclusión de un estudio especializado de un tema en particular. Por lo común la tesis se presenta al terminar los estudios, pero se puede empezar  a escribir en cualquier momento de los mismos.

Esto deja entender que una dedicatoria de tesis, está dirigida a aquellos que de alguna forma han sido de ayuda o apoyo para quien la escribe, por cualquier clase de apoyo moral, material, económico o didáctico, proporcionados. Ya sean familiares, amigos, profesores o compañeros, y  aquellos que hayan intervenido de alguna forma en el desarrollo de las ideas y experiencias que tengan como conclusión la culminación de la tesis en cuestión y a los que el autor de la misma pretenda agradecer por medio de la dedicatoria, como en el caso de maestros, compañeros, asesores, e impresores, que  de alguna forma contribuyeron a la culminación de la tesis, así como a los sinodales por aprobarla, quienes califican si en la tesis se cumplieron los objetivos de la misma, como, cuál es el problema planteado en la tesis y cuál  la propuesta de solución al problema,  qué estuvo mal en la metodología y si fueron corregidos o rectificados por la tesis en cuestión.

Generalmente las dedicatorias son breves, no más de dos o tres líneas, aunque también hay dedicatorias muy largas \cite{Dedicatoria}.

\text
\hfill \break

\noindent \textsc{Vázquez González Erick}\\

\noindent \textsc{ergovazquez@esimez.mx}\\ % URL

\noindent \textsc{Escuela Superior de Ingeniería Mecánica y Eléctrica, Unidad Zacatenco}\\

\noindent Ultima Revisión \today{} % Printing/edition 

\newpage
\thispagestyle{empty}

\section*{Introducción} % El * omite la sección del índice 

\textit{Una breve introducción que explique el por que se esta desarrollando el trabajo Así como, empezar a definir, el Producto Principal del Proyecto  de Tesis y empezar a concebir el Objetivo o Alcance Principal del Proyecto, lo cual, como se comentó antes, “es mejorar la calidad del medio ambiente”} \cite{Tesis}.


\section*{Antecedentes}

\textit{Aquí el alumno tiene que manifestar, en palabras cotidianas, todo lo que conoce sobre el tema que pretende desarrollar, de la manera más amplia y completa posible. Es deseable que indique sus referencias sobre el tema, las teorías, los conceptos y los conocimientos implicados en éste, así como la bibliografía y todos los posibles detalles que permitirán evaluar la profundidad de su conocimiento sobre el tema. Es indispensable tomar en cuenta el estado del arte de la temática propuesta.}

\textit{El objetivo de esta sección del proyecto es conocer qué tanto sabe y entiende del tema el alumno, para contar con elementos de juicio sobre sus posibilidades de desarrollo de la investigación propuesta.}

\section*{Alcances}

\textit{En este apartado del proyecto el alumno debe establecer hasta dónde llegará con su estudio, qué tanto pretende abarcar y qué dejará sin examinar en su investigación. Se trata de que el propio alumno delimite las fronteras de su trabajo, de manera que no rebase lo esperado, pero tampoco se quede por debajo de los límites establecidos.}

\textit{Es necesario señalar que no todos los temas propuestos son factibles de investigar, ya sea teórica o empíricamente; por eso hay que establecer los límites dentro de los cuales se puede actuar.}

\section*{Justificación}

\textit{Expresar, en palabras sencillas y de manera breve, las razones por las cuales desea investigar ese tema de tesis, ya sea personales, académicas, profesionales o de otra índole. }

\textit{Para el asesor es quizá el elemento más importante de la propuesta, ya que le permitirá evaluar la posibilidad de que el alumno lleve a término su investigación, en tanto que podrá valorar sus limitaciones, alcances, disponibilidad y conocimientos, entre otros muchos aspectos.}

\newpage

\section*{Objetivos}

\textit{Este apartado expresa, en palabras llanas y simples, cuál será el fin último que se pretende alcanzar con la tesis. El objetivo se determina respondiendo a estas interrogantes: ¿qué se quiere hacer?, ¿qué se pretende alcanzar?, ¿cómo se puede analizar? y ¿qué se quiere demostrar?}

\subsection*{Objetivo general}

\textit{Inherentes con un Sistema, que se desarrollará como parte del Proyecto completo.}

Generar un sistema de monitoreo eléctrico con la capacidad de medir la potencia real de una instalación eléctrica con precisión.

\subsection*{Objetivos específicos}

\textit{La suma de los objetivos específicos debe dar el objetivo global o general.}

\begin{itemize}

	\item Conocer el medio y métodos de medición del consumo eléctrico.
	\item Diseñar un sistema que permita medir y procesar la potencia real de una instalación monofásica.
	\item Construir un \gls{SIBC} para almacenar dichos parámetros. 
	\item Medir la corriente de la instalación con un bajo margen de error con forme a la norma Norma Oficial Mexicana 001-SEDE-2012. 

\end{itemize}
  

\section*{Resumen} 

\textit{Se presenta un pequeño resumen que permita al lector recopilar toda la información del protocolo para así poder proceder al primer capitulo de la tesis. }

\section*{Resume} %Se recomienda presentar el resumen en ingles

\textit{Es una copia en ingles de la sección anterior con el propósito de permitir que lectores de otros países encuentren el documento.}

\newpage