% Capítulo II Marco Contextual

%El marco contextual contiene el contexto en el que se esta llevando acabo la tesis (el país, dispositivos similares al que se desarrolla, tipos de sistemas involucrados, normatividades gubernamentales) esto para tener un marco de referencia sobre el cual trabajar los siguientes capítulos.

\chapter{ MARCO CONTEXTUAL }

\textit{El marco contextual es el medio ambiente en donde se desarrolla o desarrollará el proyecto de tesis; el Medio Ambiente Temporal o Histórico; que incluye, sus antecedentes históricos o de creación, y el Medio Ambiente Espacial o Físico, que incluye, la ubicación física de la empresa o área.}

\textit{ Todo esto, sirve para ubicar al lector (y al mismo alumno), en los antecedentes del: Marco Contextual del Medio Ambiente (empresa, área, institución), en donde se desarrollará el proyecto de tesis.}


\section{ Sistemas de monitoreo de energía }

El avance de la tecnología hace que cada vez se desarrollen nuevos sistemas que monitorizan el consumo energético, sin embargo ninguno de ellos cumple con las características necesarias para ser implementados en México esto debido al sistema de cobro implementado en el país. 

\section{ Medición y cobro de la energía en México }

A la fecha, existen en México 43 tarifas distintas para el suministro y venta de energía eléctrica clasificadas de acuerdo con su uso y su nivel de tensión. El esquema tarifario eléctrico que controla la Secretaría de Hacienda, ha privilegiado los subsidios cruzados \footnote{Este es un ejemplo de pie de pagina se utiliza para clarificar las dudas que puedan surgir en el lector a lo largo de la lectura } 

\subsection{ Cables al interior de la instalación eléctrica }

Los calibres dependen de la carga a alimentar, el calibre mínimo a utilizarse es No. 12 \gls{AWG}. Para alimentación exclusiva de lámparas puede utilizarse calibre No. 14 \gls{AWG}. Si es un solo circuito y existe una carga mayor a 3,500 Watts utilizar preferentemente calibre No. 10 para alimentadores principales. Diámetro de la tubería mínimo de 3/4 de acuerdo con la normatividad de la \gls{CFE}

\begin{table}[H]
\centering
\begin{tabular}{||c c c c c||} 
 \hline
 \multicolumn{5}{|c|}{Amperaje que soportan los cables de cobre} \\
 \hline
 mm$^2$  & AWG & Amperaje 60c& Amperaje 75c & Amperaje 90c \\ [0.5ex] 
 \hline\hline
 2.08 & 14 & 15 A& 15 A & 15 A  \\ 
 3.31 & 12 & 20 A & 20 A & 20 A \\
 5.26 & 10 & 30 A& 30 A & 30 A \\
 8.37 & 8 & 40 A& 50 A & 55 A \\
 \hline
\end{tabular}

\caption{Calibres estándar para instalaciones domesticas de acuerdo a su temperatura máxima }
\label{table:cablibres}
\end{table}

\section{ Resumen del capitulo 2 }

Resumen del capitulo e introducción al siguiente capitulo. 