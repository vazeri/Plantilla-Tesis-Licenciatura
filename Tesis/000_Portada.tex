%%%                                 INICIA 
\begin{center}																		%%%
\newcommand{\HRule}{\rule{\linewidth}{0.6mm}}									%%%\left
 																					%%%
\begin{minipage}{0.48\textwidth} \begin{flushleft}
\includegraphics[scale = 0.25]{Tesis/Imagenes/IPN}
\end{flushleft}\end{minipage}
\begin{minipage}{0.48\textwidth} \begin{flushright}
\includegraphics[scale = 0.23]{Tesis/Imagenes/Esime_Logo_Modificado.png}
\end{flushright}\end{minipage}

\pagestyle{empty} % No headers
													 								%%%
\vspace*{-1.5cm}								%%%
																					%%%	
\textsc{\LARGE  INSTITUTO POLITÉCNICO NACIONAL}\\[1.5cm]	

\textsc{\LARGE Escuela Superior de Ingenier\'ia\\ Mecánica y Eléctrica\\ \vspace{8px} \Large Unidad Profesional Zacatenco}\\[.15cm]				

\vspace{8px} \small  INGENIERÍA EN COMUNICACIONES Y ELECTRÓNICA \\[1.5cm]				
%%%

\begin{minipage}{0.9\textwidth} 

\begin{center}																					%%%
\textsc{\huge TRABAJO TERMINAL } % Se convierte en tesis de acuerdo a la evaluación de los sinodales en el examen profesional

%El  Trabajo de Grado debe tener un componente investigativo, que consiste en la formulación, planeación y en algunos casos, ejecución de un trabajo o proyecto en el que el estudiante ponga en práctica los conocimientos adquiridos en el transcurso del programa académico.
 
%De esta manera, el trabajo de grado se convierte en una oportunidad para la fundamentación, aplicación y producción de conocimientos, que conjuguen las habilidades investigativas con los saberes y competencias adquiridas a través de la formación académica y profesional, y a partir de los cuales se planteen soluciones a los problemas de su contexto social y laboral.

\end{center}

\end{minipage}\\[0.5cm]
%%%
    																				%%%
 			\vspace*{1cm}																		%%%
																					%%%
{ \huge \bfseries Guía, esquema, cuerpo del escrito 
para trabajo de grado de licenciatura\\[0.03cm] } 

% La elección del tema de tesis debe ser congruente con la carrera que se cursa, (con el mapa curricular o área de estudio) Sera evalúa por un jurado, debe buscar la solución de una problemática real, y debe ser pertinente con los estándares de la escuela )

% La tesis de licenciatura debe cubrir la suma de técnicas o modificación de una, es una propuesta de soluciones a una problematica real.

% El titulo tentativo del trabajo, esta en constante cambio conforme al desarrollo del trabajo, no te enamores de un solo titulo ni te preocupes tanto por que sea el adecuado, ya que al final del trabajo, los asesores que te hagan tus revisiones, terminan modificandolo. 	

\vspace*{1.5cm}%%%
							%%%
 								%%%
\begin{minipage}{0.46\textwidth}													%%%
\begin{flushleft} \large															%%%
P R E S E N T A N:\\	

\vspace*{.4cm}%%%

Autor 1\\     % Se permiten de 1 a 3 compañeros en un solo equipo para el desarrollo de la tesis 
Autor 2\\
Autor 3\\
\space
%%%
			%
            \vspace*{1.2cm}	
            													%%%
										 						%%%
\end{flushleft}																		%%%
\end{minipage}		
																%%%
\begin{minipage}{0.52\textwidth}		
\vspace{-0.6cm}											%%%
\begin{flushright} \large															%%%
A S E S O R E S: \\		
\vspace*{.4cm}%%%

Asesor Metodológico\\    % El asesor metodológico es el profesor asignado en la materia de DPP, (No se puede cambiar)
Asesor Técnico 1\\       % Los asesores técnicos, pueden ser elegidos por los autores de la tesis de la especialidad 
Asesor Técnico 2\\       % o de una especialidad ajena en a la que se esta cursando (Tu los elijes)

% Es importante destacar que los asesores ganan puntos por cada trabajo de grado que pasa como tesis, entre mas puntos acumulan tienen acceso a mejores prestaciones económicas por parte de la escuela, existen profesores caza puntos, que lo único que buscan es ser anotados como asesores y se esconden o simplemente no ayudan a los alumnos en sus trabajos de tesis, así que debes asegurarte de que tus asesores sean personas integras, con conocimientos que puedan ayudarte en el desarrollo de tu tesis. 

% El trabajo del asesor metodológico es darte a entender cual es el formato del trabajo, el propósito y forma de la tesis, apoyarte con tus exposiciones, y procurar que este bien redactado. Usualmente no se meten en la parte técnica.

% El trabajo de los asesores técnicos es procurar verificar que lo que estés haciendo técnicamente, a ellos se les suele entregar una copia de tus avances del escrito para que la revisen y te hagan anotaciones, correcciones de gramática y ortografía, así como también es parte de su labor apoyarte y buscar la manera de defender tu tesis ante los profesores que serán los sinodales que te evaluaran en la defensa de tu examen profesional 

\end{flushright}																	%%%
\end{minipage}	
\vspace*{1cm}

\begin{center}	

\textsc{\LARGE para obtener el titulo de:}\\[.5cm]		

 		\large{\textbf{\Large Ingeniero en comunicaciones y electrónica}	}\ % Cambiar si corresponde														%%%
\vspace{.5cm} 																				
																				
{\large México, D.F. \monthyeardate \today }	%Actualiza la fecha automáticamente, el día no debe ir en la portada solo el mes y año												%%%
 			\end{center}\end{center}		
 			 			\vspace{.1cm} 	
																		
%%%%%%%%%%%%%%%%%%%% TERMINA PORTADA %%%%%%%%%%%%%%%%%%%%%%%%%%%%%%%%