%Capítulo IV Caso Práctico/Construcción

%; Normalmente el cuerpo principal del Proyecto y documento de la tesis.  Supongamos que; se define qué: el Producto Principal a desarrollar es: Un Sistema de Información Basado en Computadoras [1], y se elige la Metodología de Leopoldo Galindo Soria (LGS), ya qué es la más cercana o la más conocida en ese medio ambiente.  A partir de ahí, se siguen las actividades que sugiere ésta Metodología, para  crear o proponer o construir el Producto Principal del Proyecto de Tesis.

\chapter{ DISEÑO DEL SISTEMA }

\textit{En este capitulo se cubre la implementación del sistema en su conjunto así como los pasos a seguir a grandes rázgos sin entrar en muchos detalles sobre como se diseña el sistema a implementar }

\section{ Requisitos funcionales del sistema }

\begin{itemize}

\item \textbf{Comunicación Inalámbrica:} Debe tener la capacidad de transmitir de manera inalámbrica la medición tomada por el micro controlador

\item \textbf{Restricción de sustancias peligrosas:} La tarjeta debe cumplir con las normas ambientales internacionales (RoHS\footnote{RoHS (De las siglas en inglés Restriction of Hazardous Substances) se refiere a la directiva 2002/95/CE de Restricción de ciertas Sustancias Peligrosas en aparatos eléctricos y electrónicos, adoptada en febrero de 2003 por la Unión Europea}) y evitar el uso de sustancias toxicas en sus materiales y fabricación.
 
\item \textbf{Capacidad de múltiples lecturas:} Se requiere una tarjeta que tenga la capacidad de medir de al menos una fase del sistema eléctrico.

\item \textbf{Lectura del consumo energético:} El usuario podrá colocar un sensor de corriente de manera segura que permita medir la intensidad que circula por el conductor.

\item \textbf{Creación de una cuenta web:} El usuario podrá crear una cuenta en la plataforma web del producto.

\end{itemize}

\section*{ Nivel usuario }

\begin{itemize}

\item Medir el consumo de la instalación eléctrica.
\item Procesar la información.
\item Interpretar la información por medio software en un navegador web.
\item Mostrar la información al usuario de una manera agradable e intuitiva.
\end{itemize}

\section*{ Nivel técnico }

\begin{itemize} 

\item Medir la corriente que fluye por el sistema eléctrico.
\item Procesar la señal por medio de un micro controlador.
\item Elegir un medio de trasmisión para la información.
\item Enviar la información a un servidor web.
\item Simplificar la información de manera gráfica y clara para el usuario.

\end{itemize}


\section*{Nivel sistema}

\begin{itemize} % [noitemsep] removes whitespace between the items for a compact look

\item Acondicionar las señales para el Convertidor Analógico Digital del microcontrolador.
\item Aplicar un corrimiento de \gls{CC} para que el micro controlador pueda leer $2^8$ valores (256).
\item Canalizar por un puerto del micro controlador la señal digitalizada.
\item Enviar la señal de manera inalámbrica a través de un sistema de radio frecuencia.
\item Procesamiento de los datos de entrada del lado servidor.
\item Visualización de los datos a través de una aplicación web.
\end{itemize}


%Capítulo IV Caso Práctico/Construcción

