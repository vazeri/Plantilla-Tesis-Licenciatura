\chapter{ ANÁLISIS, EVALUACIÓN Y DIAGNOSTICO }

\textit{Este capitulo permite establecer un diagnostico general de los sistemas existentes, sobre los cuales se desarrollara el tema de tesis pretende abarcar.}

\section{ Análisis de la situación actual }

\begin{itemize}
	\item  El usuario no entiende de manera clara cuanto es lo que deberá pagar en moneda nacional.

    \item La compañía cambia anualmente los precios de sus tarifas por \gls{KWh}.
    
    \item Dependiendo del consumo existirá o no un subsidio en el recibo por parte del gobierno.
    
\end{itemize}

\section{ Evaluación }

    \textit{Es necesario conocer los dispositivos o sistemas similares al que se pretende desarrollar en este ejemplo se analizan el medidor analógico y el digital}

\subsection{ Medidor analógico mecánico }
    
\subsection{ Medidor inteligente digital }

\section{ Ventajas y desventajas de los sistemas actuales }
    
    La siguiente tabla contiene los sistemas de medición, que se utilizan actualmente para el cobro de la energía eléctrica en México.
    
    \begin{table}[H]
    \centering
    \begin{tabular}{| m{3.8cm} | m{7.3cm}| m{6.2cm} |} 
    \hline
    
    Tipo de Medidor & Ventajas & Desventajas\\
    \hline\hline
    Medidor Digital & Auto protegidos, sensibles a alteraciones, desconexiones, medición digital, Comunicación vía linea  eléctrica hacia la compañía & No ofrece por si mismo Información relevante al usuario , costo elevado para la compañía, desconexión remota de energía. \\
    \hline
    Medidor Analógico (Descontinuado)& Medición por medio de inductores, Mecanismo Mecánico & Fácilmente alterables, ya no están en producción ni distribución\\
    \hline
    \end{tabular}
     \caption{Tabla con los sistemas actuales de medición de energía domestica en México}
     \label{tblsistemas}
    \end{table}

\section{ Diagnostico }

\textit{Utilizando el análisis, ahora es posible establecer un diagnostico de la problemática que se pretende resolver con base en el análisis realizado de los sistemas existentes.}

\section{ Resumen del capitulo 3 }

\textit{Resumen del capitulo e introducción al siguiente capitulo.}