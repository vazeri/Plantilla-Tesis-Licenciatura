% Es la identificación de todos los conceptos y términos clave que se utilizarán para el desarrollo de la tesis. Lo importante es que el alumno exprese e identifique, con sus propias palabras, las principales definiciones que conoce del tema propuesto y que están relacionadas con su tema. Exprese esos términos como los conoce y los entiende. De esta manera, se podrá evaluar más objetivamente su conocimiento real sobre el tema.

\newglossaryentry{ADC}
{
  name=ADC,
  description={(Convertidor Analógico Digital)  Del inglés "Analog-to-Digital Converter",es un dispositivo electrónico capaz de convertir una señal analógica de tensión en una señal digital con un valor binario}
}

\newglossaryentry{formula}
{
    name=formula,
    description={A mathematical expression}
}

\newglossaryentry{SIBC}
{
  name=SIBC,
  description={(Sistema de Información Basado en Computadoras) Es la integración de hardware, software, personas, procedimientos y datos. Todos estos elementos se conjugan, trabajando juntos, para proporcionar información básica para la toma de decisiones con mayor calidad y facilidad.  }
}

\newglossaryentry{CFE}
{
  name=C.F.E.,
  description={(Comisión Federal de Electricidad)  es una empresa productiva del Estado, encargada de controlar, generar, transmitir y comercializar energía eléctrica en todo el territorio mexicano }
}

\newglossaryentry{CC}
{
  name=C.C.,
  description={ (Corriente Continua)  se refiere al flujo continuo de carga eléctrica a través de un conductor entre dos puntos de distinto potencial, que no cambia de sentido con el tiempo. [ Electrotecnia, ciclos formativos. Peter Bastian]
}
}

\newglossaryentry{CA}
{
  name=C.A.,
  description={ (Corriente Alterna) a la corriente eléctrica en la que la magnitud y el sentido varían cíclicamente. La forma de oscilación de la corriente alterna más comúnmente utilizada es la oscilación senoidal con la que se consigue una transmisión más eficiente de la energía [La corriente alterna, IES, Gerald Brenan.]
}
}

\newglossaryentry{HTML}
{
  name=HTML,
  description={ Lenguaje de marcas de hipertexto (por sus siglas en ingles), es lenguaje para la elaboración de páginas web es un estándar que sirve de referencia para la elaboración de páginas web en sus diferentes versiones, define una estructura básica para la definición de contenido de una página web]
}
}

\newglossaryentry{FP}
{
  name=F.D.P.,
  description={ (Factor De Potencia) Es la relación entre la potencia activa, y la potencia aparente, Da una medida de la capacidad de una carga de absorber potencia activa.[Antonio Gómez Expósito; Fundamentos de teoría de circuitos.]
}
}

\newglossaryentry{FIDE}
{
  name=FIDE,
  description={ (Fideicomiso para el Ahorro de Energía Eléctrica) Fideicomiso privado, sin fines de lucro, constituido el 14 de agosto de 1990, por iniciativa de la Comisión Federal de Electricidad (CFE), en apoyo al Programa de Ahorro de Energía Eléctrica; para coadyuvar en las acciones de ahorro y uso eficiente de la energía eléctrica. 
}
}

\newglossaryentry{PROFECO}
{
  name=PROFECO,
  description={ (Procuraduría Federal del Consumidor) organismo público descentralizado e independiente de la Secretaría de Economía del Gobierno Federal Mexicano. Fue creado para promover y proteger los derechos del consumidor, fomentar el consumo inteligente y procurar la equidad y seguridad jurídica en las relaciones entre proveedores y consumidores
}
}

\newglossaryentry{DAC}
{
  name=D.A.C,
  description={ (Tarifa de alto consumo) Tarifa doméstica de alto consumo, es una multa que cobra la CFE por exceder un límite de consumo en KWh (kilowatt-hora) en su hogar.
}
}

\newglossaryentry{DBMS}
{
  name=DMBS,
  description={ (Sistema de Administración de Bases de Datos) Sistema de administración de bases de datos. Software que controla la organización, almacenamiento, recuperación, seguridad e integridad de los datos en una base de datos}
}

\newglossaryentry{SGBD}
{
  name=SGBD,
  description={ (Sistema de Gestion de Bases de Datos) conjunto de programas que permiten el almacenamiento, modificación y extracción de la información en una base de datos, además de proporcionar herramientas para añadir, borrar, modificar y analizar los datos.}
}

\newglossaryentry{SQL}
{
  name=SQL,
  description={ (Lenguaje de consulta estructurado) por sus siglas en inglés Structured Query Language es un lenguaje declarativo de acceso a bases de datos relacionales que permite especificar diversos tipos de operaciones en ellas.}
}

\newglossaryentry{IP}
{
  name=IP,
  description={ (Dirección IP) Es una etiqueta numérica que identifica, de manera lógica y jerárquica, a una interfaz (elemento de comunicación/conexión) de un dispositivo (habitualmente una computadora) dentro de una red que utilice el protocolo IP (Internet Protocol), que corresponde al nivel de red del modelo OSI}
}

\newglossaryentry{WiFi}
{
  name=WiFi,
  description={ Mecanismo de conexión de dispositivos electrónicos de forma inalámbrica.}
}

\newglossaryentry{PCB}
{
  name=PCB,
  description={ (Placa de Circuito Impreso) Superficie constituida por, pistas o buses de material conductor laminadas sobre una base no conductora. El circuito impreso se utiliza para conectar eléctricamente a través de las pistas conductoras.}
}

\newglossaryentry{TVS}
{
  name=TVS,
  description={ (Supresores para transitorios de voltaje) Diodo usado para proteger componentes electrónicos de descargas electrostáticas causadas por picos de voltaje inducidos en cables o pistas de circuitos [ What are TVS diodes, Semtech Application Note SI96-01] }
}

\newglossaryentry{NOM}
{
  name=NOM,
  description={ (Normatividad Mexicana) Serie de normas cuyo objetivo es regular y asegurar valores, cantidades y características mínimas o máximas en el diseño, producción o servicio de los bienes de consumo entre personas morales y/o personas físicas, sobre todo los de uso extenso y de fácil adquisición por parte del público en general [Asociación de Normalización y Certificación, 2015] }
}

\newglossaryentry{IEEE}
{
  name=IEEE,
  description={ (Instituto de Ingeniería Eléctrica y Electrónica) Asociación mundial de técnicos e ingenieros dedicada a la estandarización y el desarrollo en áreas técnicas. [ IEEE at a Glance 2011] }
}

\newglossaryentry{ESD}
{
  name=ESD,
  description={ (Descarga electrostática) es un fenómeno electrostático que hace que circule una corriente eléctrica repentina y momentáneamente entre dos objetos de distinto potencial eléctrico; como la que circula por un pararrayos tras ser alcanzado por un rayo se utiliza para describir las corrientes indeseadas momentáneas que pueden causar daño al equipo electrónico. [Fundamentals of Electrostatic Discharge, An Introduction to ESD, 2013, ESD Association] }
}

\newglossaryentry{AVR}
{
  name=AVR,
  description={ Familia de microcontroladores RISC del fabricante estadounidense Atmel. El acrónimo AVR no significa nada en particular deacuerdo a ATMEL  [Embedded C Programming and the Atmel AVR, Larry O'Cull, 2006] }
}

\newglossaryentry{LGS}
{
  name=LGS,
  description={ Metodología de desarrollo de sistemas de información basados en computadora  [Herrera Martínez Alejandro ] }
}

\newglossaryentry{AWG}
{
  name=AWG,
  description={ (Calibre de alambre estadounidense) Referencia de clasificación de diámetros para conductores eléctricos (cables) indicados con la referencia AWG. Cuanto más alto es este número, más delgado es el alambre. El alambre de mayor grosor (AWG más bajo) es menos susceptible a la interferencia, posee menos resistencia interna y, por lo tanto, soporta mayores corrientes a distancias más grandes.  [Servicios Condumex (2005) Manual técnico de cables de energía ] }
}


\newglossaryentry{RMS}
{
  name=RMS,
  description={ (Valor Cuadrático Medio) Denominado valor eficaz. Se define como el valor de una corriente rigurosamente constante (corriente continua) que al circular por una determinada resistencia óhmica pura produce los mismos efectos caloríficos (igual potencia disipada) que dicha corriente variable (corriente alterna)  [Fundamentos de Circuitos Eléctricos, Charles K. Alexander, 2012] }
}


\newglossaryentry{CFL}
{
  name=CFL,
  description={ (Lámpara Fluorescente Compacta)  es un tipo de lámpara que aprovecha la tecnología de los tradicionales tubos fluorescentes para hacer lámparas de menor tamaño que puedan sustituir a las lámparas incandescentes con pocos cambios en la armadura de instalación y con menor consumo   }
}

\newglossaryentry{UART}
{
      name=UART,
  description={ (Transmisor-Receptor Asíncrono Universal) es el dispositivo que controla los puertos y dispositivos serie. Se encuentra integrado en la placa base o en la tarjeta adaptadora del dispositivo.  [Alternativas al bajo consumo CFL] }
}


\newglossaryentry{ENIG}
{
      name=ENIG,
  description={ (Electrolítico de oro de inmersión de níquel) Tipo de recubrimiento utilizado para placas de circuito impreso. Se compone de un recubrimiento de níquel no electrolítico cubierto con una fina capa de oro de la inmersión, que protege el níquel de la oxidación. [ "Surface Finishes in a Lead Free World, Uyemora International 2014] }
}
\newglossaryentry{IDE}
{
      name=IDE,
  description={(Integrated Development Environment) Es un ambiente de desarrollo interactivo o entorno de desarrollo integrado  es una aplicación de software, que proporciona servicios integrales para facilitarle al programador de computadora el desarrollo de software }
}

\newglossaryentry{TDH}
{
      name=TDH,
  description={ (Distorsión armónica total) Es la relación entre el contenido armónico de la señal y la primera armónica o fundamental. La distorsión armónica total nunca debe estar por encima del 1\%. De estarlo, en lugar de enriquecer la señal, la distorsión empieza a desvirtuarla }
}

\newglossaryentry{KWh}
{
      name=KWh,
  description={ (Killowatt Hora) Es una unidad de energía equivalente a 1000 watts por hora o 3.6 mega joules si la energía es transmitida o usa a un ritmo constante  (potencia) sobre un periodo de tiempo la energía total es el producto de la potencia por el tiempo , usualmente se utiliza como unidad de cobro para la energía consumida por usuarios de instalaciones eléctricas ["Thompson, Ambler and Taylor, Barry N. (2008). Guide for the Use of the International System of Units (SI) "] }
}

\newglossaryentry{watt}
{
      name=watt,
  description={  El vatio o watt es la unidad de potencia del Sistema Internacional de Unidades. Su símbolo es W. Expresado en unidades utilizadas en electricidad, un vatio es la potencia eléctrica producida por una diferencia de potencial de 1 voltio y una corriente eléctrica de 1 amperio (1 voltiamperio). ["Diccionario de la lengua española (22.ª edición), Real Academia Española, 2001, consultado el 21 de marzo de 2015."] }
}

\newglossaryentry{LED}
{
      name=LED,
  description={ (Light-Emitting Diode) ‘diodo emisor de luz’; es un componente optoelectrónico pasivo y, más concretamente, un diodo que emite luz.}
}

\newglossaryentry{VAR}
{
      name=VAR,
  description={ (Volt Amper Reactivo) Potencia que no posee un carácter realmente de ser consumida; sólo aparece cuando hay bobinas o condensadores en los circuitos. Dicha potencia tiene un valor medio nulo, por lo que no produce trabajo necesario. 
  }
  }


\newglossaryentry{MOS}
{
      name=MOS,
  description={ MOS Technology, Inc., también conocida como $Commodore Semiconductor Group$, (al ser adquirida por CBM), fue un fabricante de calculadoras y microprocesadores, siendo famosa por su microprocesador MOS Technology 6502. [Información sobre los chips MOS y su uso en los ordenadores – Ronald van Dijk]
}
}

\newglossaryentry{USB}
{
      name=USB,
  description={ ( Universal Serial Bus) El Bus Universal en Serie (BUS), más conocido por la sigla USB, es un bus estándar industrial que define los cables, conectores y protocolos usados en un bus para conectar, comunicar y proveer de alimentación eléctrica entre computadoras, periféricos y dispositivos electrónicos. [Boston Globe Online  USB deserves more support. Simson.net. 31 de diciembre de 1995]
  }
  }

\newglossaryentry{FTDI}
{
      name=FTDI,
  description={ (Future Technology Devices International), comúnmente conocida por su abreviatura FTDI, es una empresa privada escocesa de dispositivos semiconductores, especializada en tecnología Universal Serial Bus. }
}
